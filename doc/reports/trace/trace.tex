\documentclass[12pt]{article}
                                % Don't waste so much paper!
\usepackage[height=9.0in,letterpaper]{geometry}

\usepackage{times}
\usepackage{mathptmx}

\usepackage{amsmath,amssymb}    % for math
\usepackage{subfigure,graphicx} % for figures
\usepackage{natbib}		% bibliography
\usepackage{appendix,fancyhdr}  % nice formatting
\usepackage{longtable}          % for the class list


\usepackage{hyperref}
% \hypersetup{dvips,colorlinks=true,allcolors=blue}
% \hypersetup{dvips}
\hypersetup{
  dvips,
  bookmarks=true,         % show bookmarks bar?
  unicode=false,          % non-Latin characters in Acrobat’s bookmarks
  pdftoolbar=true,        % show Acrobat’s toolbar?
  pdfmenubar=true,        % show Acrobat’s menu?
  pdffitwindow=false,     % window fit to page when opened
  pdfstartview={FitH},    % fits the width of the page to the window
  pdftitle={My title},    % title
  pdfauthor={Author},     % author
  pdfsubject={Subject},   % subject of the document
  pdfcreator={Creator},   % creator of the document
  pdfproducer={Producer}, % producer of the document
  pdfkeywords={keyword1} {key2} {key3}, % list of keywords
  pdfnewwindow=true,      % links in new window
  colorlinks=false,       % false: boxed links; true: colored links
  linkcolor=red,          % color of internal links (change box color with linkbordercolor)
  citecolor=green,        % color of links to bibliography
  filecolor=magenta,      % color of file links
  urlcolor=cyan           % color of external links
}
\usepackage{hypcap}
\usepackage{caption}

%% Add a trick to get around the figure float stuff

\makeatletter                   % Change the catagory code of @
\newenvironment{inlinetable}{%
  \def\@captype{table}%
  \noindent\begin{minipage}{0.999\linewidth}\begin{center}\footnotesize}
    {\end{center}\end{minipage}\smallskip}

\newenvironment{inlinefigure}{%
  \def\@captype{figure}%
  \noindent\begin{minipage}{0.999\linewidth}\begin{center}}
    {\end{center}\end{minipage}\smallskip}
\makeatother                    % Revert to protected

%% For convenience . . . 

\newcommand*{\myprime}{^{\prime}\mkern-1.2mu}
\newcommand*{\mydprime}{^{\prime\prime}\mkern-1.2mu}
\newcommand*{\mytrprime}{^{\prime\prime\prime}\mkern-1.2mu}

\newcommand{\bx}{{\bf x}}
\newcommand{\bt}{{\bf\theta}}
\newcommand{\bp}{{\bf\pi}}
\newcommand{\barm}{{\bar m}}
\newcommand{\bv}{{\bf v}}
\newcommand{\va}{\bv_a}
\newcommand{\vb}{\bv_b}
\newcommand{\vap}{\va\myprime}
\newcommand{\vbp}{\vb\myprime}

% \lta and \gta produce > and < signs with twiddle underneath
\def\spose#1{\hbox to 0pt{#1\hss}}
\newcommand{\lta}{\mathrel{\spose{\lower 3pt\hbox{$\mathchar"218$}}
    \raise 2.0pt\hbox{$\mathchar"13C$}}}
\newcommand{\gta}{\mathrel{\spose{\lower 3pt\hbox{$\mathchar"218$}}
    \raise 2.0pt\hbox{$\mathchar"13E$}}}

\let\lesssim=\lta
\let\gtrsim=\gta

\newcommand{\msun}{{\rm\,M_\odot}}
\newcommand{\AU}{\,\hbox{\rm AU}}
\newcommand{\kms}{{\rm\,km\,s^{-1}}}
\newcommand{\pc}{\,\hbox{\rm pc}}
\newcommand{\kpc}{{\rm\,kpc}}
\newcommand{\Myr}{\,\hbox{\rm Myr}}
\newcommand{\Byr}{\,\hbox{\rm Byr}}
\newcommand{\Gyr}{\,\hbox{\rm Gyr}}
\newcommand{\gyr}{\,\hbox{\rm Gyr}}
% \newcommand{\mod}{\mathop{\rm mod}\nolimits}
\newcommand{\sgn}{\mathop{\rm sgn}\nolimits}
\newcommand{\ext}{{\hbox{\ninerm ext}}}
\newcommand{\resp}{{\hbox{\ninerm resp}}}
\mathchardef\star="313F
\newcommand{\bth}{{\bf\theta}}
\newcommand{\bdata}{{\bf D}}
\newcommand{\bdth}{{\Delta{\bf\theta}}}
% \newcommand{\Re}{\mathop{\it Re}\nolimits}
% \newcommand{\Im}{\mathop{\it Im}\nolimits}
\newcommand{\Ei}{\mathop{\rm Ei}\nolimits}

\newcommand{\apj}  {ApJ}
\newcommand{\apjl} {ApJL}
\newcommand{\apjs} {ApJS}
\newcommand{\aj}   {AJ}
\newcommand{\mnras}{MNRAS}

\newcommand{\model}{{\cal M}}
\newcommand{\data} {{\bf D}}
\newcommand{\param}{\boldsymbol{\theta}}

%%% 
%%% End definitions
%%% 

\pagestyle{fancy}

\setlength{\headheight}{15pt}
\pagestyle{fancy}

% \renewcommand{\chaptermark}[1]{\markboth{#1}{}}
% \renewcommand{\sectionmark}[1]{\markright{#1}{}}
 
\fancyhf{}
\fancyhead[LE,RO]{\thepage}
\fancyhead[RE]{\textit{\nouppercase{\leftmark}}}
\fancyhead[LO]{\textit{\nouppercase{\rightmark}}}
 
% \shorttitle{Trace species}
% \shortauthors{Weinberg}

\begin{document}

\title{Notes on implementing the trace-species algorithm for EXP}

\author{Martin D. Weinberg\\
  Department of Astronomy\\
  University of Massachusetts, Amherst, MA, USA 01003;\\
  {\it weinberg@astro.umass.edu}
}

\date{\today}

\maketitle

\tableofcontents

\section{Introduction}
\label{sec:intro}

These notes summarize my current progress to date on specifying a
procedure for handling trace species.

\section{Collision pairs}
\label{sec:background}

Here, I derive the probability of interaction for simulation particles
is derived from the true physical cross section.

The DSMC procedure can be viewed and examined from many different
viewpoints, e.g. as a stochastic process that mimics the Boltzmann
equation or by adopting a more physical viewpoint, as we choose to do
here, as a computational gas composed of computational particles. The
properties of the computational gas such as its average number density
over a specified region of space differ from those of the actual
gas. That is, all quantities related to the computational gas have to
be suitably rescaled to obtain those of the physical gas that is being
modeled. 

The
collision frequency of a single particle of the physical gas of
species a with particles of gas of species b at a number density of n
b is given by: 
\begin{equation}
  \nu_{ab} = n_b \overline{\sigma_{ab} c_r}, 
\end{equation}
where $c_r$ is the relative velocity between particles and
$\sigma_{ab}$ is the cross section for elastic collisions between
species a and b. The bar over a quantity designates its average value
in the ensemble of all sample particles. The properties of the DSMC
gas differ from those of the physical gas and are denoted with a
tilde. In the simulation, species b has a number density of
\begin{equation}
n_b = \Gamma_p W_b {\tilde n}_b
\end{equation}
where $\Gamma_p$ is the fiducial number of true particles per
computational particle and $W_b$ is the weighting relative to some
fiducial particle.  That is, each computational particle represents
$W_b\Gamma_p$ actual particles. Similarly, the single particle
collision rate of the numerical gas is denoted by ${\tilde\nu}_{ab}$.

The particles of the numerical gas have a collisional cross section
that is different than those of the actual physical particles. It is
denoted by ${\tilde\sigma}_{ab}$ to distinguish it from $\sigma_{ab}$:
\begin{equation}
  \tilde{\nu}_{ab} = \tilde{n}_b \overline{\tilde{\sigma}_{ab} \tilde{c}_r}.
\end{equation}
The DSMC procedure is based on the fundamental assumption that all
particle-carried quantities (e.g. velocity, position, internal energy)
are identical between the physical and computational gases. They thus
share the same velocity, so that ${\tilde c}_r = c_r$ as well as the
same collision rate which implies that $\tilde{\nu}_{ab} =
\nu_{ab}$. Putting these equations together we have
\begin{equation}
  n_b \overline{\sigma_{ab} c_r} =
  \frac{n_b}{W_b\Gamma_p}\overline{\tilde{\sigma}_{ab} \tilde{c}_r}.
  \label{eq:balance}
\end{equation}
That is, the computational cross
section ${\tilde\sigma}$ is given by: 
\begin{equation}
{\tilde\sigma}_{ab} = W_b\Gamma_p \sigma_{ab}.
\end{equation}

The probability of a collision, $P_{coll}(a_i, b_j)$, between a moving
particle $a$ and a target particle $b$ belonging, respectively, to
species $a$ and $b$, both located inside the same cell of volume V can
be interpreted as being given by the ratio of the volume of the
cylinder engendered by the collision cross section through the
relative motion of both particles, $c_r(a_i, b_j)$, during time
$\Delta\tau$ to the volume of the cell:
\begin{equation}
  P_{coll}(a_i, b_j) = \frac{{\tilde\sigma}_{ab}(a_i, b_j)c_r(a_i, b_j)\Delta\tau}{V}.
\end{equation}
Collisions between particle pairs in a cell at a given time step are
independent events, that is the probability of a collision occurring
for a given $\{(a_i, b_j\}$ pair is independent of those of all other
pairs in the cell. Thus, the number of collisions that occurs during a
time step is obtained by summing the probability of occurrence of each
potential collision. When $a\not=b$, $N_a\times N_b$ collisions may
occur in principle, so that
\begin{equation}
N_{coll}(a, b) = \sum_{i_a=1}^{N_a}\sum_{j_b=1}^{N_b} P_{coll}(i_a, j_b) =
\sum_{i_a=1}^{N_a}\sum_{j_b=1}^{N_b}\frac{\tilde{\sigma}_{ab}(i_a, j_b)
  c_r(i_a, j_b)\Delta\tau}{V},
\end{equation}
and the average value of the cross section times the velocity over the
ensemble of the $N_a\times N_b$ pairs is defined as:
\begin{equation}
  \overline{\tilde{\sigma}_{ab} c_r}\equiv
  \frac{1}{N_a N_b} \sum_{i_a=1}^{N_a}\sum_{j_b=1}^{N_b} \tilde{\sigma}_{ab}(i_a, j_b)
  c_r(i_a, j_b).
\end{equation}

In the case when collisions between particles of the same species are
considered, that is when $a = b$, only $\binom{N_a}{2} = N_a(N_a-1)/2$
collision pairs are possible so that $N_{coll}(a, a)$ becomes
\begin{equation}
  N_{coll}(a, a) = \sum_{i_a=1}^{N_a}\sum_{j_a=i_a+1}^{N_a} P_{coll}(i_a, j_a) =
  \sum_{i_a=1}^{N_a}\sum_{j_a=i_a+1}^{N_a}\frac{{\tilde\sigma}_{aa}(i_a, j_a)
    c_r(i_a, j_a)\Delta\tau}{V}
\end{equation}
and
\begin{equation}
  \overline{\tilde{\sigma}_{aa} c_r} \equiv
  \frac{2}{N_a (N_a-1)}
  \sum_{i_a=1}^{N_a}\sum_{j_a=i_a+1}^{N_a} {\tilde\sigma}_{aa}(i_a, j_a)
  c_r(i_a, j_a).
\end{equation}

Using these equations, a combined expression for $N_{coll}$ for both
$a=b$ and $a\not=b$ may be written
\begin{equation}
  N_{coll}(a, b) = \frac{1}{1+\delta_{ab}}
  \frac{\overline{\tilde{\sigma}_{ab}c_r}\Delta\tau}{V} N_a(N_b - \delta_{ab})
\end{equation}
or
\begin{equation}
  N_{coll}(a, b) = \frac{N_a W_b\Gamma_p(N_b - \delta_{ab})}{(1+\delta_{ab})V}
  \sigma_{ab}c_r\Delta\tau.
\end{equation}

This gives an expression for the number of collisions that has to
occur during $\Delta\tau$ between the particles of the numerical gas
so that equation (\ref{eq:balance}) is satisfied. To increase
computational efficiency, the DSMC procedure considers potential
collisions with a greater number of particle pairs, i.e.
\[
\frac{(\sigma_{ab}c_r)_{max}}{(\sigma_{ab}c_r)} \ge 1
\]
times the $N_{coll}(a, b)$ number of collision pairs, denoted by
$N_{pairs}(a, b)$:
\begin{equation}
  N_{pairs}(a, b) = \frac{(\sigma_{ab}c_r)_{max}}{(\sigma_{ab}c_r)} N_{coll}(a, b)
\end{equation}
or
\begin{equation}
  N_{pairs}(a, b) = \frac{N_a W_b\Gamma_p(N_b - \delta_{ab})}{(1+\delta_{ab})V}
  (\sigma_{ab}c_r)_{max}\Delta\tau.
\end{equation}

The collision probability between particles a and b of species a and b
now may be written as following
\begin{equation}
  P_{coll}(i_a, j_b) = \frac{\sigma_{ab}(i_a, j_b) c_r(i_a,
    j_b)}{(\sigma_{ab}c_r)_{max}},
\end{equation}
so that $N_{coll}$ is correctly obtained on average over all collision
pairs. In the case where $W_a = W_b$, the number of pairs to consider
for collision has to be altered as the velocity of both particles
cannot simultaneously change to conserve average momentum during the
collision. This issue is addressed in the next two sections.

\section{Collision dynamics}

In the absence of species relative weights and in the actual physical
gas, the properties of both participating particles change after a
collision. In the numerical gas, however, when particles with
different relative weights participate in a collision, their
properties cannot both equally be affected by it. If they were, the
overall translational energy of particles and momentum would not be
conserved on average in the gas and the effect of collisions on the
species with the largest weight from that with the lowest would be
systematically overestimated. In the following, we present the
collision scheme that is most commonly used to handle collisions
between particles with different relative weights which was first
introduced. Because the scheme does not conserve energy, an
energy-conserving collision scheme was subsequently proposed by
Boyd. The latter was for instance later used in ref. where a detailed
description of the impact of relative weights on the modeling of
chemical reactions via DSMC is given. An energy conserving scheme is,
however, of limited utility for the test case that will later be
studied as the residence time of particles inside the domain is
relatively short with very few undergoing multiple collisions and is
therefore not implemented.  Non-energy conservation is, however, much
more important for closed systems, such as homogeneous heat baths
where boundary conditions are not constantly resupplying new particles
(and thus energy).

Considering a collision between a particle of species a with mass
$m_a$ and relative weight $W_a$ and one of species b with mass $m_b$
and relative weight $W_b$, the pre-collision velocities of both
particles are designated by $\va$ and $\vb$ and their post-collision
velocities are designated by $\vap$ and $\vbp$. 

For elastic collisions and equal weight particles (i.e. $W_a=W_b$), we
may write:
\begin{eqnarray}
  \vap &=& \bv_m + \frac{m_b}{m_a+m_b}\bv_r \\
  \vbp &=& \bv_m - \frac{m_a}{m_a+m_b}\bv_r
\end{eqnarray}
where the center of mass velocity is $\bv_m$ and $\bv_r$ is the
post-collision relative velocity in the center-of-mass frame.  For
$W_a\not=W_b$, 

Now, let us define a split and merge scheme for interactions between
species of types a and b; we assume that $W_b < W_a$ and split a
particle of species a into two component with weight $q\equiv W_b/W_a$
and $1 - W_b/w_a = 1 - q$, respectively.  We collide the smaller
component of type a with the full particle b and then recombine,
conserving momentum.

Conservation of the overall momentum during the collision
requires
\begin{equation}
  m_a W_a \va + m_ b W_b \vb = m_a W_a \vap + m_b W_b \vbp.
  \label{eq:consmom}
\end{equation}
This may be rewritten by splitting as follows:
\begin{eqnarray}
  m_a W_a \va + m_ b W_b \vb &=& W_a\left[ m_a (1-q) \va + q(m_a\va +
    m_b \vb) \right] \\ \label{eq:precoll}
  &=& W_a\left[ m_a (1-q) \va + q(m_a\va\mydprime +
    m_b \vb\mydprime) \right], \label{eq:postcoll}
\end{eqnarray}
where the double-primed quantities in equation (\ref{eq:postcoll})
indicate post-collision velocities.  Identifying terms in equations
(\ref{eq:consmom}) and (\ref{eq:postcoll}) yields:
\begin{eqnarray}
  \vap &=& (1-q)\va + q\va\mydprime \\
  \vbp &=& \vb\mydprime
\end{eqnarray}

Unfortunately, one does not conserve momentum and energy
simultaneously.  After some algebra it, one may show that the
difference between the pre-collision and post-collision energies is:
\begin{equation}
  \Delta E = \frac12 W_a m_a q(1-q)(\va - \vap)^2
\end{equation}
assuming that the collision is elastic. Since we have assumed that
$W_a>W_b$, we must have $0<q<1$ and therefore $\Delta E \ge 0$ which
implies that energy is always lost.  Standard practice (Boyd, ref) is
to replace this energy in the center of mass of the next non-trace
collision.

\section{Energy loss owing to inelastic collisions}

Although there is some flexibility in this procedure, it seems natural
to remove the energy from the center of mass of the collision.  As in
previous implementations, the deficit can be accumulated if there is
insufficient energy.

\section{Assigning particle masses and weights}

\subsection{Overview}

To maintain the relative kinematics of the different species, the
superparticle mass ratios should be preserved.  If hydrogen and helium
are species a and b, then $m_b/m_a = m_{He}/m_H \approx 4$.

Let $f_X$ denote the mass fraction of species $X$ and $N_X$ be the
number of superparticles of species $X$.  The total mass of each
species is:
\begin{equation}
  M_X = \Gamma_p \sum_{j=1}^{N_X} m_{X,j} W_{X,j} = 
  \Gamma_p N_X \overline{m}_{X} \overline{W}_{X},
\end{equation}
where 
\begin{eqnarray}
  \overline{W}_X &\equiv& \frac{\sum_{j=1}^{N_X} m_{X,j}
    W_{X,j}}{\sum_{j=1}^{N_X} m_{X,j}} \label{eq:meanW} \\
  \overline{m}_X &\equiv& \frac{\sum_{j=1}^{N_X} m_{X,j}}{N_X}.
  \label{eq:meanM}
\end{eqnarray}
More typically, we might have $\overline{m}_{X} = m_{X,j}$ and $\overline{W}_{X}
= W_{X,j}$ and not need the mean quantities in equations
(\ref{eq:meanW}) and  (\ref{eq:meanM})

The mass true mass fraction constraint implies:
\begin{equation}
\frac{f_Y}{f_X} = \frac{N_Y \overline{m}_{Y} \overline{W}_{Y}}{N_X \overline{m}_{X} \overline{W}_{X}}
\end{equation}
or
\begin{equation}
\frac{\overline{W}_{Y}}{\overline{W}_{X}} = 
\left(\frac{N_X \overline{m}_{X}}{N_Y \overline{m}_{Y}}\right)
\left(\frac{f_Y}{f_X}\right).
\end{equation}
If superparticles of all species are distributed equally by number
than $N_X=N_Y$ and
\begin{equation}
\frac{\overline{W}_{Y}}{\overline{W}_{X}} = 
\left(\frac{\overline{m}_{X}}{\overline{m}_{Y}}\right)
\left(\frac{f_Y}{f_X}\right).
\end{equation}
For example, assuming that $Y=\mbox{He}$ and $X=\mbox{H}$, we have
$\overline{W}_Y/\overline{W}_X \approx 0.087$.

\subsection{Implementation}

In EXP, the superparticles are assumed to have their true masses in
system units.  I will denote these as $\hat{m}$.  Then
\begin{equation}
  \frac{f_Y}{f_X} = \frac{N_Y {\hat m}_{Y}}{N_X \hat{m}_{X}} = \frac{{\hat m}_{Y}}{\hat{m}_{X}}
\end{equation}
where the last equality obtains for $N_X=N_Y$.  However, each particle
has its own number weighting so that
\begin{equation}
\frac{{\hat m}_{Y}}{\hat{m}_{X}} = \frac{\overline{m}_{Y} \overline{W}_{Y}}{\overline{m}_{X} \overline{W}_{X}}
\end{equation}
or
\begin{equation}
\frac{\overline{W}_{Y}}{\overline{W}_{X}} = 
\left(\frac{{\hat m}_{Y}}{\hat{m}_{X}}\right)
\left(\frac{\overline{m}_{X}}{\overline{m}_{Y}}\right).
\end{equation}

To get individual masses from system masses, recall that $\Gamma_p$ is
the conversion of system mass to atomic mass units.  Then $\overline{m}_X =
\hat{m_x}m_a\Gamma_p \overline{W}_X$ where $m_a$ is the mean atomic mass unit.

\bibliography{master}
\label{sec:ref}

\end{document}

